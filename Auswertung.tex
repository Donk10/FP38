\documentclass[12pt, english]{scrartcl} %Koma-Klasse "Artikel" mit 12pt Schrift
\usepackage[left=20mm, right=20mm, top= 25mm, bottom=25mm]{geometry}  % Seitenrändern
\usepackage{graphicx} %zum Einfügen von Bildern
\usepackage[english]{babel} %englische Silbentrennung
\usepackage{multicol}
\usepackage{graphicx}
\usepackage{caption}
\graphicspath{/home/michael/Downloads/eigene Auswertung/graphics}
\usepackage{siunitx}

\sisetup{separate-uncertainty}
\usepackage{braket}

\title{F18/38 Atmospheric Spectroscopy}
\author{Carsten L{\"u}th \and Michael Dorkenwald}
\date{\today}

\begin{document}
\maketitle

\begin{multicols}{2}


\section{Abstract}
Lorem Ipsum.
\section{Introduction}




\section{Background}

\subsection{Ozone and Nitrogendioxid}

\subsection{The DOAS Measurement System}

\newpage

\section{Active Measurement of a Nitrogendioxid gas-cell}
In this part of the experiment we measured the absorption of a $NO_2$ gas-cell in the spectrum of an Hg-lamp. Due to the nature of this measurement we could simplify the problem by using the fact that we have access to $I_0$ by taking a spectrum without the gas cell in the light path and then take a measurement with the gas-cell in the light path to measure $I$. After dark current and offset were corrected for both measurements we used the simplified lambert-beer law to get to equation.
\begin{equation}
\tau = \log(\frac{I_0(\lambda)}{I_0(\lambda)})= \sigma_{NO_2} \cdot \rho \cdot L
\end{equation}
The reference convolution for $NO_2$ was then used to compute the SCD$= \rho \cdot L$. This was then used to compute the density $\rho = (2.81 \pm 0.07 ) \cdot 10^{-7} \frac{\text{mol}}{\text{cm}^3}$ and the mixture ratio $(6.29 \pm 0.39) \cdot 10^3 \text{ppm}$.
\section{Sunlight Measurement of a recorded daycycle}

\section{Sunlight Measurement }



\end{multicols}

\end{document}