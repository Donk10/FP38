\documentclass[12pt, english]{scrartcl} %Koma-Klasse "Artikel" mit 12pt Schrift
\usepackage[left=20mm, right=20mm, top= 25mm, bottom=25mm]{geometry}  % Seitenrändern
\usepackage{graphicx} %zum Einfügen von Bildern
\usepackage[english]{babel} %englische Silbentrennung
\usepackage{multicol}
\usepackage{graphicx}
\usepackage{caption}
\graphicspath{/home/michael/Downloads/eigene Auswertung/graphics}
\usepackage{siunitx}

\sisetup{separate-uncertainty}
\usepackage{braket}

\title{F18/38 Atmospheric Spectroscopy}
\author{Carsten L{\"u}th \and Michael Dorkenwald}
\date{\today}

\begin{document}
\maketitle

\begin{multicols}{2}


\section{Abstract}
In this experiment the trace gases Nitrogendioxid ($NO_2$) and Ozone ($O_3$) are examined with DOASIS. These gases play an important role for life on earth. Therefore it is important to have a method to determine the concentration in the air. The DOASIS (DOAS Intelligent System) was developed at the Institut for Environmental physics at University Heidelberg
\section{Introduction}

explain how they effect our live on earth 
First, a Nitrogendioxid gas-cell is explored 
Second, 

\section{Background}

\subsection{Ozone and Nitrogendioxid}

\subsection{The DOAS Measurement System}

\newpage

\section{Active Measurement of a Nitrogendioxid gas-cell}


\section{Sunlight Measurement of a recorded daycycle}

\section{Sunlight Measurement different evaluations angles}

\end{multicols}

\end{document}